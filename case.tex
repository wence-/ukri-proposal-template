% -*- TeX-engine: luatex -*-
\documentclass[
squeezespace=false,
showframe=false,
a4paper,
fontsize=11pt]{ukriproposal}

\usepackage[l2tabu, orthodox]{nag}

% biblatex setup
\usepackage[
% dois
doi=true,
% no arxiv id
eprint=false,
% No URLs
url=false,
% numeric
style=authoryear,
% Sort by name then year
sorting=nyt,
defernumbers=true,
% no isbn
isbn=false,
% Max names in text citation
% maxcitenames=3,
% Max names in reference section
% maxbibnames=1,
% Use biber not bibtex for citation management
backend=biber,
% A.N. Author
giveninits=true,
uniquename=init]{biblatex}
% If you need to cull parts of the bibliography entries...
% \AtEveryBibitem{\clearfield{title}}
% \AtEveryBibitem{\clearfield{pages}}
\setlength{\bibitemsep}{0pt}

% This defines a new "check" command to the printbibliography so that
% we can have separate bibliographies but the papers are unique across
% the document. From
% https://tex.stackexchange.com/a/517234
\usepackage{bibonce}

% Use this bibliography
% \addbibresource{refs.bib}

% citet/citep are natbib specific.
\def\citet{\textcite}
\def\citep{\parencite}

\usepackage{cleveref}

\begin{document}

% This introduces a segment of the document in which we're gathering
% references
\begin{refsegment}

\maketitle

\section*{Track record}

\section*{Our code}

\subsection*{References}
% Print the bibliography for this segment
\printbibliography[heading=none,
resetnumbers=false,
segment=\therefsegment,
check=onlynew]
% We're done gathering
\end{refsegment}

\newpage

\maketitle

\begin{refsegment}
% Methodology and approach.

% Risks and mitigation strategies.

\section{Introduction and project aims}

\subsection*{Further references}
% Print the bibliography for this segment
% check=onlynew means we don't print things we saw in the previous
% segments.
\printbibliography[heading=none,
resetnumbers=false,
segment=\therefsegment,
check=onlynew]
\end{refsegment}
\end{document}
